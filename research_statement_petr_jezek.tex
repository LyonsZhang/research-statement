%\documentstyle[11pt,a4]{article}
%\documentclass[a4paper]{article}
\documentclass[a4paper, 10pt]{article}
% Seems like it does not support 9pt and less. Anyways I should stick to 10pt.
%\documentclass[a4paper, 9pt]{article}
\topmargin-2.0cm

\usepackage{fancyhdr}
\usepackage{pagecounting}
\usepackage[dvips]{color}

% Color Information from - http://www-h.eng.cam.ac.uk/help/tpl/textprocessing/latex_advanced/node13.html

% NEW COMMAND
% marginsize{left}{right}{top}{bottom}:
%\marginsize{3cm}{2cm}{1cm}{1cm}
%\marginsize{0.85in}{0.85in}{0.625in}{0.625in}

\advance\oddsidemargin-0.95in
%\advance\evensidemargin-1.5cm
\textheight9.2in
\textwidth6.75in
\newcommand\bb[1]{\mbox{\em #1}}
\def\baselinestretch{1.15}
%\pagestyle{empty}

\newcommand{\hsp}{\hspace*{\parindent}}
\definecolor{gray}{rgb}{0.4,0.4,0.4}
%\definecolor{gray}{rgb}{1.0,1.0,1.0}


\begin{document}
\thispagestyle{fancy}
%\pagenumbering{gobble}
%\fancyhead[location]{text}
% Leave Left and Right Header empty.
\lhead{}
\rhead{}
%\rhead{\thepage}
\renewcommand{\headrulewidth}{0pt}
\renewcommand{\footrulewidth}{0pt}
\fancyfoot[C]{\footnotesize \textcolor{gray}{}}

%\pagestyle{myheadings}
%\markboth{Petr Je\v{z}ek}{Petr Je\v{z}ek
\pagestyle{fancy}
\lhead{\textcolor{gray}{\it Petr Je\v{z}ek}}
\rhead{\textcolor{gray}{\thepage/\totalpages{}}}
%\rhead{\thepage}
%\renewcommand{\headrulewidth}{0pt}
%\renewcommand{\footrulewidth}{0pt}
%\fancyfoot[C]{\footnotesize http://www.stanford.edu/$\sim$sundaes/application}
%\ref{TotPages}

% This kind of makes 10pt to 9 pt.


%\vspace*{0.1cm}
\begin{center}
{\LARGE \bf RESEARCH STATEMENT}\\
\vspace*{0.1cm}
{\normalsize Petr Je\v{z}ek (jezekp@kiv.zcu.cz)}
\end{center}
%\vspace*{0.2cm}

%\begin{document}
%\centerline {\Large \bf Research Statement for Petr Je\v{z}ek}
%\vspace{0.5cm}

% Write about research interests...
%\footnotemark
%\footnotetext{Check This}



% Say that research work has been both theoritical and practical.

\subsection*{Introduction}

My research interests lead in the area of Electroencephalography and Event Related Potentials (EEG and ERP). In particular, my research widens management of data and metadata of EEG/ERP experiments. A common thread of my research is a definition of the domain and its brief description together with the development of a strategy for sharing data, metadata, experimental protocols or signal processing methods and its results among interested laboratories. My research combines knowledge from areas of databases, information retrieval, domain ontologies, Semantic Web technologies, Big Data an so no. My research has been published in several papers cited in well-known databases including Thomson Reuters, Scopus or IEEE.

\subsection*{Background and Current Work}

\subsubsection*{EEG/ERP Experimental Data and Metadata}

The scientific efforts focused on EEG/ERP experiments usually describe designing and creating experimental scenarios or performing experiments. They usually do not solve storage, interchange or description of experimental data and medatata.

EEG and ERP experiments take usually long time and produce a lot of data. With the increasing number of experiments it is necessary to solve their long-term storage and management. There is a sort of problems: (1) A widely spread and generally used standard for EEG/ERP data files within the community does not exist. (2) Results of EEG/ERP experiments have the same importance as obtained data (when experiments have to be repeated the results and input data have to be available). (3) There is no reasonable and easily extensible tool for long-term EEG/ERP data (metadata) storage and management (the general practice is to store data and metadata in private not accessible repositories). (4) There is no common practice to share and interchange data between EEG/ERP laboratories.



\subsubsection*{Semantic Web and Common Data Modeling}

When the experimental data are obtained they should be clearly interpreted. The unification of data/metadata formats relates to the need to interchange of experimental data. The Internet seems to be an appropriate medium to share experimental data. However, due to limits that the current web gradually reaches a parallel web called the Semantic Web is being developed. The current web consists of large collections of data without any classification. The Semantic Web that expresses meaning of data by domain ontologies is supposed to solve the problem of missing semantics of the current web.

Development of specific ontologies is crucial task in the creation of the Semantic Web. Ontologies usually serve as recognizable data sources accessible by automatic software readers. However, current software systems are usually based on object-oriented programming languages and they operate over large data collections.  The data layer of such systems is usually represented by the set of data objects. These objects are most often stored within the relational database. The mapping of relational schema (RDB) to common object oriented modeling (OOM) is solved quite satisfactorily using e.g Hibernate Framework. In contrast, mapping of OOM to Semantic Web technologies is not solved yet due to diverse expressivity of both domains. The solving of this issue is desirable. 

Since fundamental differences between semantics of the object-oriented code and Semantic Web languages exist, it is necessary to ensure a suitable mapping.

Because expressive capabilities of Semantic Web languages are richer than in the case of the object-oriented code I proposed a one way mapping that fill these semantic gaps.

I have developed an java-based approach where I extend a common JavaBean using Java Annotations that add missing semantics into the plain JavaBean \cite{Jezek10}. I presented the defined mapping of Java Annotations to the corresponding OWL constructs \cite{DBLP:conf/iceis/JezekM11}. I also presented the developed framework where the mapping is implemented \cite{DBLP:conf/biostec/JezekM12}. This framework is supposed to be a powerful tool for preparing domain ontologies extracted from object-oriented code of Java-based systems independently on the specific domain needs.


\subsubsection*{EEG/ERP Portal}

As the members of the Czech National Node (CNNN) of International Neuroinformatics Coordinating Facility (INCF) we participate in definition and development of standardized formats
for electrophysiology research. Our efforts, following INCF recommendations, resulted in the development of the ontology that describes metadata. These metadata are organized in
several semantic groups (experimental scenario, experimenters and tested subjects, used hardware, description of raw data, etc.) This ontology was presented to large neuroscience community (through presented papers and my doctoral thesis) to accept it as the standard ontology of EEG/ERP experiments.

Within CNNN we developed a solution called the EEG/ERP Portal \cite{DBLP:conf/biostec/JezekM10}. The EEG/ERP Portal serves the community as the practical tool for storing, managing and interchanging custom experiments. The internal structure of the system is designed to satisfy restrictions given by the ontology. It is also developed with regards to a need to store a large collection of experiments. I suppose it as an initial proposal to remove mentioned difficulties.

The developed Semantic Framework was integrated \cite{DBLP:conf/biostec/MoucekJP11} with the EEG/ERP Portal, therefore we are able to transform the portal ontology into the Semantic Web languages (OWL, RDF). This ontology was registered in the  Neuroscience Information Framework (NIF) \cite{DBLP:conf/bmei/JezekM11} developed by USA INCF Node. This approach proved the correctness of our solution.

\subsubsection*{External Modules}

Since CNNN is responsible for ensuring  whole infrastructure for EEG/ERP research including not only data collection, but also support for signal processing or reproducing experiments according to experimental scenarios we prepared a complete software infrastructure integrated with the EEG/ERP Portal \cite{DBLP:conf/ic3k/MoucekJJP11, Overview-of-neuroinformatics-infrastructure-in-Pilsen}.

The modules are implemented outside systems and placed into the open source repositories. The modules are divided into the two functional groups. The first one includes modules integrated in the EEG/ERP Portal directly. It includes: \textit{Semantic Framework} - it ensures generating of OWL documents from stored experiments. \textit{Waveforms Reader} - it enables visualization of stored data. \textit{Analytic Methods} - implementation of methods for signal processing as ICA, Fast ICA, Matching Pursuit, Wavelet transform or Huang-Hilbert method. The second one represents stand-alone systems that communicate with the portal via Web Service API. It includes: \textit{JERPA} \cite{DBLP:conf/biostec/JezekM11} - it is a desktop tool for visualizing EEG/ERP records. \textit{Insomnia - EEG/ERP Processor} - it is a system running on the separate server where are mentioned analytical methods installed as plug-ins. It is developed with regard to the need to handle many requests simultaneously. \textit{Library of Statistical Methods} - including statistical methods as one way and two way analysis of variance (ANOVA) and one way and two way multivariate analysis of variance (MANOVA). It runs on the separate server as well. \textit{Brain Recorder} - it is a software ensures communication with analogue-digital converter. \textit{Annotation Tool} - it is used by the domain expert when preparing annotated model of data that are transformed into the output OWL document.



\subsection*{Future Plans}

\subsubsection*{Context}

Basically storing of experimental data is very important assets when experiments are performed. When such data are described by a suitable metadata an incredible amount of knowledge can be derived from them. These data can be harvested and analyzed by various interested researchers over the world. With sharing of data relates a sharing of partial research result together with possibilities to share implemented signal processing methods.

We particularly covered mentioned difficulties by proposing a custom infrastructure described in previous sections. The presented approach works quite satisfactorily but improvements in performance, data security, or responsibility are still under the development. Each experiment contains a raw data their size is tens megabytes of half-hour experiment and tens of largely heterogenous metadata items. With increasing count of experiments amount of database space is enormous.

Because of a raising popularity of cloud solutions I would like to investigate possibilities that individual cloud solutions provide with regards to needs of storing neurophysiological data. In addition, with sharing of the data closely relates sharing of knowledge between laboratories. I would like to use selected social networks that could be a suitable medium for sharing knowledge and experimental results.

\subsubsection*{Cloud Computing in Neurophysiology Domain}

When Cloud Computing is an emerging trend related with management of Big Data I would like to explore its usability for sharing of neurophysiological data. I plan to build a system based on current developed EEG/ERP Portal. The system will be inspired by existing approached as e.g. Carmen Portal or the systems developed by J-Node od Neuroinformatics but completely cloud-based.  I would like to design a whole infrastructure that will be prepared for storing and sharing data, experimental protocols or signal processing methods with ensuring a high level of security. Because experiments contain personal data a privacy of sensitive data has to be ensured. For example, if a user puts a custom experimental data there will be a process that ensures anonymization of stored data. It means that other user will be able to download raw data with metadata including information about e. g. an experimental scenario but not e. g. about a tested subject name.

The system should enable management of custom experiments with any specific computer skills because it is supposed to be used not only by informatics. I suppose usage of common web based interface available for every common equipped client's computer. When many experiments are also performed outside the home laboratory I plan to provide also an interface for mobile devices. The data collected via a mobile device will be synchronized with the data stored within the base cloud system.

I suppose that the architecture of proposed system will contain several submodules where each will be responsible for a different activity that a user can request. The proposed modules are: \textit{Database Cloud Service} - A storage for experimental data/metadata. \textit{Core Cloud Service} - It will take the responsibilities for management of the system. It will include the submodules for management of data/matadata upload, download, service management, metadata management, system security or management of workflows. \textit{Semantic Web Cloud} - It will responsible for a Semantic Web representation of stored data. \textit{Web Service Cloud} - It will responsible for communication with external systems. \textit{Social Network Cloud} - It will communicate with social networks or analyze of social networks comments.

Since the Semantic Web output (OWL) will be available for advanced users I would like to investigate the design of a search interface within extensible OWL documents sets . This interface should be accessible by automatic readers (software agents) that will be able to infer semantic relationships from provided OWL document.

It should be noted that there is necessary to investigate who will be responsible for the data stored within the cloud and who will manage the system. My initial idea is either to find a suitable open source solution or better INCF seems to be an appropriate authority. Because this task is important it will be  necessary to pay it careful attention.

\subsubsection*{Connection with Data from Social Networks}

When popularity of social networks  (e. g. Facebook, Twitter or LinkedIn) is raising amount of provided information is increasing. Such networks should be suitable medium for sharing information between researchers. I plan to provide an internal social network that will be integrated within the cloud-based system. This internal social network will enable to system users creating a custom user groups. The internal social network will be connected with external social networks (e. g. Facebook, Twitter or LinkedIn). When a user publish a comment to an internal social network it will be also possible to publish it on a selected external social network.

In addition, a back synchronization is desirable. When a user is logged on external social network he/she can download comments into the internal social network. But the question is how to profit from the comments of not internal system users. The challenge is the automatic extraction of information from the external social networks comments. My plan is to develop a framework that will automatically browse social networks and find comments according to defined rules. The next big challenge is exploitation these obtained information to data analysis. Finally, the scalable solution that handle this large data collection is desirable.

The last substantial task in this part of research is studying the security and privacy of published information on social networks. For example if a comment is posted on a social network and these information is re-used by other users the question is what to do with the personal data of the original author. It will need to investigate legal or ethical aspects of sharing data in common and especially using the public social networks.

\subsection*{Conclusion}

My current and planned future research combines an investigation of dealing with experimental data from neurophysiological research and usability of Cloud Computing for this domain. I have proposed the EEG/ERP Portal that particularly solves handling of a large set of experimental data but does not solve precisely a collection of heterogenous data, data security or responsibility for the data. The data are searchable using the Semantic Web technologies that are directly supposed for reading by automatic readers.

Because of a number of obtained data is still increasing the management of such sources is increasingly difficult. As a suitable solution I suppose developing a cloud-based solution. I am going to investigate usability, suitability and security of such solution with respects security and privacy of stored experiments. The data are also supposed to be synchronized with several cooperating systems e.g. mobile devices or Rich Clients.

The last step is the development of framework for analyzing comments from social networks and its synchronization with the internal social network integrated with the system. The social network comments should be precisely analyzed in order to required knowledge could be extracted.

The aim of the presented research statement is mainly providing an infrastructure for EEG/ERP research. Nevertheless I strongly believe that the results of current and future work are applicable on other domains, because of investigated models, technologies and approaches are based on the general question.


%\newpage

%\begin{thebibliography}{deSolaPITH}
% Change font size?
% \tiny, \footnotesize, \small,\normalsize, \large, \Large, \LARGE, and \huge
%\begin{small}
\bibliography{bibtex}{}
\bibliographystyle{plain}


\end{document}

